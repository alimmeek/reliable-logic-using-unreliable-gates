\documentclass[tikz]{standalone}
\usepackage[calc]{datetime2}
%    the pgfgantt documentation is here: 
%    https://ctan.org/pkg/pgfgantt
\usepackage{pgfgantt}
\usepackage{datenumber}

\title{CS352 Term 1 Gantt Chart}
\author{Ian Saunders}
\DTMsavedate{StartOfTerm}{2023-10-02}

\begin{document}

% ------ date operations -----
%set the date for gantt bars to start at
\newcounter{datecounter}
\newcommand{\setTheDate}[1]{
  \setmydatenumber{datecounter}{\DTMfetchyear{StartOfTerm}}{\DTMfetchmonth{StartOfTerm}}{\DTMfetchday{StartOfTerm}}%
  \DTMsavedate{periodEnd}{#1}
  \setmydatenumber{x}{\DTMfetchyear{periodEnd}}{\DTMfetchmonth{periodEnd}}{\DTMfetchday{periodEnd}}%
  \addtocounter{x}{-\thedatecounter}% 
}

% a command to add some days to a date and save it
\newcount\daycount
\newcommand{\saveStartDatePlusDays}[2]{
    \DTMsaveddateoffsettojulianday{StartOfTerm}{#1}\daycount
    \DTMsavejulianday{#2}{\number\daycount}
}

% command to make a bar that follows the previous one
\newcounter{x}
\newcommand{\addDays}[1]{
    \addtocounter{x}{#1}
 }
\newcommand{\nextganttbar}[2]{
    \saveStartDatePlusDays{\arabic{x}}{dfrom}
    \addDays{#2}
    \addtocounter{x}{-1}
    \saveStartDatePlusDays{\arabic{x}}{dto}
    \addtocounter{x}{1}
    \ganttbar{#1}{\DTMusedate{dfrom}}{\DTMusedate{dto}}
}

% ------ setup date range for chart -----
\saveStartDatePlusDays{-1}{prevDate}
\saveStartDatePlusDays{73}{endOfTerm} % count 9 weeks
\newcommand{\startDate}{\DTMusedate{StartOfTerm}}
\newcommand{\prevDate}{\DTMusedate{prevDate}}


% ------ setup and styling for the chart -----
\begin{ganttchart}[hgrid, vgrid, inline,
    bar/.append style={fill=blue!25},
    time slot format=isodate %YYYY-MM-DD
]{\startDate}{\DTMusedate{endOfTerm}}

% command to create a label on the left. 
\newcommand\ganttlabel[1]{
    \ganttbar[inline=false]{#1}{\startDate}{\prevDate}
}

% command to make an orange bar
\newganttchartelement{barorange}{
    barorange/.append style={fill=orange!80 },
}

% titles showing month date and term week (starting at week 1)
\gantttitlecalendar{month=name, day, week=1} 

\ganttnewline

\ganttnewline

% ------ plan your timeline below! -----

\ganttlabel{C351}
\ganttbarorange{Project Spec}{2023-10-03}{2023-10-12}
\ganttbarorange{Progress Report}{2023-11-13}{2023-11-27}
\\

\ganttlabel{CS352}
\ganttbarorange{Interim Report}{2023-11-02}{2023-11-16}
\ganttbarorange{Interim Report Feedback}{2023-11-16}{2023-11-23}
\ganttbarorange{Group Presentation}{2023-10-13}{2023-12-07}
\ganttbarorange{Final Report}{2023-12-04}{2023-12-14}
\\

\ganttlabel{CS325}
\ganttbarorange{Coursework}{2023-10-03}{2023-11-20}
\\

\ganttlabel{CS349}
\ganttbarorange{Essay}{2023-11-28}{2023-12-11}
\\

\ganttlabel{ES3C5}
\ganttbarorange{Coursework}{2023-10-27}{2023-11-30}
\\

\ganttlabel{Research}
\ganttbar{Initial}{2023-10-05}{2023-10-08}
\ganttbar{Lit Review}{2023-10-13}{2023-11-06}
\\
\ganttlabel{Development}
\setTheDate{2023-11-07}
\nextganttbar{Implementation using ideal gates}{21}
\\
\setTheDate{2023-11-28}
\nextganttbar{Design unreliable gates modules}{10} %make this longer due to courswork
\\  
\setTheDate{2023-12-07}
\nextganttbar{Modify initial implementation to use simulated gates}{5} 
\\

\end{ganttchart}

\end{document}