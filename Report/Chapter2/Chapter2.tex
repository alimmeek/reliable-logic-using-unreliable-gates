% !TEX root =  ../Report.tex
\section{Objectives}
\label{sec:objs}

\noindent There are many previously documented architectures for fault tolerance to be drawn from in this project. \\

\noindent The essential requirements for the project are listed below, using the MoSCoW system to allow for better prioritisation. The 
``Must'' and ``Should'' objectives are determined to be the minimum requirements in order for the goals of the project to have been 
achieved. The “Could” objectives enhance the quality of the project but are not necessary for its successful completion. A more detailed 
description of implementation is given under the Implementation section of the relevant method.

\begin{table}[H] 
\begin{tabular}{| M{0.47\textwidth} | M{0.47\textwidth} |}
    \hline
    Requirement & Priority  \\ \hline
    Research the various proposed architectures and ECCs & \textbf{\emph{{MUST}}} \\ \hline
    Select a method of fault tolerance to simulate & \textbf{\emph{{MUST}}} \\ \hline
    Design an implementation in the chosen language & \textbf{\emph{{MUST}}} \\ \hline
    Design a testbench to analyse the accuracy vs performance trade off & \textbf{\emph{{MUST}}} \\ \hline
    Vary the error probability of the gates & \textbf{\emph{{SHOULD}}} \\ \hline
    Repeat with a software method and compare their advantages and disadvantages & \textbf{\emph{{COULD}}} \\ \hline
\end{tabular}
\caption{Project requirements and their associated MoSCoW priority}
\label{table:reqs}
\end{table}

\noindent Implementation of the hardware solution to fault tolerance targets the Artix 7 xc7a100tcsg324-1 device, chosen as it allows for rapid 
prototyping and easy testing using tools with which familiarity has been gained from prior exposure through the course. The software 
solution has been tested on the Department of Computer Science`s system to ensure repeatability. 
